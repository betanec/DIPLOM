\pagestyle{empty} 
\input{title/blank-header} 
 
\begin{center} 
  {\Large{\textbf{Задание на НИР}}} 
 
  \large 
  
  Студенту гр. \theauthorgroup{} \theauthorfulldative 
\end{center} 
 
\begin{center} 
  \uppercase{\textbf{\large{}Тема НИР}} 
 
  {\Large\thetitle} 
 
  \vskip 1em 
 
  \uppercase{\textbf{\large{}Задание}} 
\end{center} 
 
{\linespread{1.0} 
  \footnotesize 
  \noindent 
  \begin{longtable}{|p{.5cm}|p{250pt}|>{\centering\arraybackslash}p{2cm}|>{\centering\arraybackslash}p{2cm}|>{\centering\arraybackslash}p{2cm}|} \hline 
  \multicolumn{1}{|>{\centering\arraybackslash}p{0.5cm}|}{№\par п/п} & \multicolumn{1}{c|}{Содержание работы} & Форма отчетности & Срок исполнения & Отметка о выполнении\par {\scriptsize Дата, подпись} \\\hline 
\projecttask & Аналитическая часть &&& \\\hline 
% (указываются предмет и цели анализа) 
\projectsubtask & Изучение и анализ существующих когнитивныйх архитектур 
  & Пункт ПЗ 
  & 
  & 
  \\\hline 
\projectsubtask & Изучить методы машинного обучения 
  & Пункт ПЗ 
  & 
  & 
  \\\hline 
\projectsubtask & Изучение и анализ проблемной области распознования речи 
  & Пункт ПЗ 
  & 
  & 
  \\\hline 
\projectsubtask & Классификация и определение эмоций 
  & Пункт ПЗ 
  & 
  & 
  \\\hline 
\projectsubtask & Оформление расширенного содержания пояснительной записки (РСПЗ) & Текст РСПЗ & 10.04.2022 & \\\hline% 
\projecttask & Теоретическая часть &&& \\\hline 
\projectsubtask & Описать модель работы виртуального актора 
  & 
  & 
  & 
  \\\hline 
\projectsubtask & Разработка алгоритмов распознования речи 
  & Алгоритмы 
  & 
  & 
  \\\hline 
\projectsubtask & Модификация модификация методов машинного обучения применитьльно к задаче распознования речи 
  & Алгоритмы 
  & 
  & 
  \\\hline 
\projecttask & Инженерная часть &&& \\\hline 
\projectsubtask & Построить модель тембора голоса 
  & Модели 
  & 
  & 
  \\\hline 
\projectsubtask & Построить семантическую модель распознования голоса 
  & Модели 
  & 
  & 
  \\\hline 
\projectsubtask & Проектирование инструментов для анализа текста 
  & Макеты 
  & 
  & 
  \\\hline 
\projectsubtask & Проектирование приложения 
  & Макеты 
  & 
  & 
  \\\hline 
\projectsubtask & Результаты проектирования оформить с помощью UML диаграмм 
  & UML диаграммы 
  & 
  & 
  \\\hline 
\projecttask & Технологическая и практическая часть &&& \\\hline 
\projectsubtask & Реализация модели виртуального актора 
  & Исполняемые файлы, исходный текст  
  & 
  & \\\hline 
\projectsubtask & Реализация программного приложения 
  & Исполняемые файлы, исходный текст  
  & 
  & \\\hline 
\projectsubtask & Реализация и дообучение моделей машинного обучения 
  & Исполняемые файлы, исходный текст  
  & 
  & \\\hline 
\projecttask & Оформление пояснительной записки (ПЗ) и иллюстративного материала для доклада. & Текст ПЗ, презентация & 06.05.2022 & \\\hline 
\end{longtable} 
} 
%\refsection 
%\nocite{Sychev} 
%\nocite{Sokolov} 
%\nocite{Gaidaenko} 
%\begin{center} 
  %\uppercase{\textbf{\large{}Литература}} 
%\end{center} 
%\printbibliography[heading=none] 
%\endrefsection 
 
\begin{center} 
  \uppercase{\textbf{\large{}Литература}} 
\end{center} 
 
\begin{itemize} 
  \item Tikhomirova, D. V., Chubarov, A. A., Samsonovich, A. V. (2019). Empirical and
  modeling study of emotional state dynamics in social videogame paradigms.
  Cognitive Systems Research.

  \item Samsonovich A.V. Comparative Analysis of Implemented Cognitive Architectures//
  Biologically Inspired Cognitive Architectures. 2011. Vol. 233. P. 469-479. doi:
  10.3233/978-1-60750-959-2-469

  \item  Larue, O., West, R., Rosenbloom, P. S., Dancy, C. L., Samsonovich, A. V.,
  Petters, D., Juvina, I. (2018). Emotion in the common model of 74 A.V.
  Samsonovich / Cognitive
  Systems Research 60 (2020) 57–76 cognition. Procedia Computer Science, 145,
  740–746.

  \item  Madl, T., Franklin, S., Chen, K., Trappl, R. (2018). A computational cognitive
  framework of spatial memory in brains and robots. Cognitive Systems Research,
  47, 147– 172.

  \item  Laird, J. E., Lebiere, C., Rosenbloom, P. S. (2017). A standard model of the
  mind: Toward a common computational framework across artificial intelligence,
  cognitive science, neuroscience, and robotics. AI Magazine, 38(4), 13–26./

\end{itemize} 
 



\vfill 
 
{\noindent\linespread{2.0}

\begin{tabularx}{\linewidth}{p{140pt}XXX} 
  Дата выдачи задания: & Студент & \hrulefill & \theauthor \\ 
  10.10.2021           &   Руководитель    & \hrulefill & \thesupervisor \\ 
\end{tabularx} 
}