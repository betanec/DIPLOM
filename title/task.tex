\pagestyle{empty}
\input{title/blank-header}

\begin{center}
  {\Large{\textbf{Задание на НИР}}}

  \large
 
  Студенту гр. \theauthorgroup{} \theauthorfulldative
\end{center}

\begin{center}
  \uppercase{\textbf{\large{}Тема НИР}}

  {\Large\thetitle}

  \vskip 1em

  \uppercase{\textbf{\large{}Задание}}
\end{center}

{\linespread{1.0}
  \footnotesize
  \noindent
  \begin{longtable}{|p{.5cm}|p{250pt}|>{\centering\arraybackslash}p{2cm}|>{\centering\arraybackslash}p{2cm}|>{\centering\arraybackslash}p{2cm}|} \hline
  \multicolumn{1}{|>{\centering\arraybackslash}p{0.5cm}|}{№\par п/п} & \multicolumn{1}{c|}{Содержание работы} & Форма отчетности & Срок исполнения & Отметка о выполнении\par {\scriptsize Дата, подпись} \\\hline
\projecttask & Аналитическая часть &&& \\\hline
% (указываются предмет и цели анализа)
\projectsubtask & Изучение и анализ классических математических моделей Земли и способов нахождения кратчайших расстояний на них
  & Пункт ПЗ
  &
  &
  \\\hline
\projectsubtask & Изучить материалы описывающие движение воздушного судна
  & Пункт ПЗ
  &
  &
  \\\hline
\projectsubtask & Изучение и анализ теории компьютерной графики применительно к построению полигональных сеток
  & Пункт ПЗ
  &
  &
  \\\hline
\projectsubtask & Анализ методов компьютерной графики применительно
к задачам построения заданных кривых в пространстве
  & Пункт ПЗ
  &
  &
  \\\hline
%\projectsubtask & Анализ методов нахождения пересечений кривых и полигональных сеток в трехмерном пространстве
  %& Пункт ПЗ
  %&
  %&
  %\\\hline
\projectsubtask & Оформление расширенного содержания пояснительной записки (РСПЗ) & Текст РСПЗ & 10.10.2021 & \\\hline%
\projecttask & Теоретическая часть &&& \\\hline
% (указываются используемые и разрабатываемые модели, методы, алгоритмы)
\projectsubtask & Аффинное преобразование
  &
  &
  &
  \\\hline
\projectsubtask & Описание математической модели Земли
  & Модели
  &
  &
  \\\hline
\projectsubtask & Разработка алгоритмов построения полигональной сетки на основе таблицы высот земной поверхности
  & Алгоритмы
  &
  &
  \\\hline
\projectsubtask & Разработка алгоритмов построения траектории движения воздушного судна
  & Алгоритмы
  &
  &
  \\\hline
\projectsubtask & Модификация методов аппроксимации компьютерной графики для построения траектории движения воздушного судна
  & Алгоритмы
  &
  &
  \\\hline
\projecttask & Инженерная часть &&& \\\hline
% (указывается, что конкретно необходимо спроектировать, а также используемые для этого методы, технологии и инструментальные средства)
\projectsubtask & Проектирование методов и алгоритмов компьютерной графики, адаптированных под задачу построения полигональной модели Земли
  & Макеты
  &
  &
  \\\hline
\projectsubtask & Проектирование классов и функций, для расчёта траектории движения воздушного судна
  & Макеты
  &
  &
  \\\hline
\projectsubtask & Проектирование классов и функций, для нахождения пересечения кривых и полигональной сетки
  & Макеты
  &
  &
  \\\hline
\projectsubtask & Проектирование юнит-тестов для проверки корректности работы статической библиотеки
  & Макеты
  &
  &
  \\\hline
\projectsubtask & Результаты проектирования оформить с помощью UML диаграмм
  & UML диаграммы
  &
  &
  \\\hline
\projecttask & Технологическая и практическая часть &&& \\\hline
% (указывается, что конкретно должно быть реализовано и протестировано, а также используемые для этого методы, инструментальные средства, технологии)
\projectsubtask & Реализация статической библиотеки адаптированных методов и алгоритмов компьютерной графики, для построения полигональной сетки
  & Исполняемые файлы, исходный текст 
  &
  & \\\hline%
\projectsubtask & Реализация статической библиотеки адаптированных методов и алгоритмов компьютерной графики, для построения траектории воздушного судна
  & Исполняемые файлы, исходный текст 
  &
  & \\\hline%
\projectsubtask & Реализация статической библиотеки адаптированных методов и алгоритмов компьютерной графики, для нахождения пересечения траектории воздушного судна и земной поверхности
  & Исполняемые файлы, исходный текст 
  &
  & \\\hline%
\projectsubtask & Тестирование статической библиотеки адаптированных методов и алгоритмов компьютерной графики
  & Исполняемые файлы, исходные тексты тестов и тестовых примеров
  &
  &
  \\\hline
\projecttask & Оформление пояснительной записки (ПЗ) и иллюстративного материала для доклада. & Текст ПЗ, презентация & 01.01.2022 & \\\hline
\end{longtable}
}
%\refsection
%\nocite{Sychev}
%\nocite{Sokolov}
%\nocite{Gaidaenko}
%\begin{center}
  %\uppercase{\textbf{\large{}Литература}}
%\end{center}
%\printbibliography[heading=none]
%\endrefsection

\begin{center}
  \uppercase{\textbf{\large{}Литература}}
\end{center}

\begin{itemize}
  \item Никулин Е. А. Компьютерная геометрия и алгоритмы машинной графики. — СПб: БХВ-2. Петербург, 2003
  \item Эдвард Энджел. Интерактивная компьютерная графика. Вводный курс на базе OpenGL = Interactive Computer Graphics. A Top-Down Approach with Open GL. — 2-е изд. — : «Вильямс», 2001
  %\item Колесников К. С. Динамика ракет / К. С. Колесников. – М. : Машиностроение, 2003
  %\item Абгарян К. А. Динамика ракет / К. А. Абгарян, И. М. Рапопорт. – М. : Машиностроение, 1969
  \item Ефремов А.В., Захарченко В.Ф., Овчаренко В.Н., Суханов В.Л. Динамика полета: учебник для студентов высших учебных заведений – М. : Машиностроение, 2011
  \item Bjarne Stroustrup The C++ Programming Language Special Edition. - М.: Издательство «БИНОМ» 2012
  \item Andrew Koenig, Barbara E. Moo Accelerated C++: Practical Programming by Example.
\end{itemize}


\vfill

{\noindent\linespread{2.0}
  \begin{tabularx}{\linewidth}{p{140pt}XXX}
    Дата выдачи задания: & Руководитель & \hrulefill & \theauthor \\
    10.10.2021           & Студент      & \hrulefill & \thesupervisor \\
  \end{tabularx}
}
