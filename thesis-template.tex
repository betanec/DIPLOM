% arara: xelatex
% arara: biber
% arara: xelatex: { synctex: true }


\documentclass[12pt,a4paper,oneside,final]{report}
\usepackage{tabularx}
\usepackage{float}
% \usepackage[russian]{babel}

\input{chapters/thesis-template-macro.tex}
\authorgroup{М20-504}
\author{Клычков М. Д.}
\authorfulldative{Клычкову Матвею Дмитриевичу}
\supervisor{Самсонович А. В.}
\consultant{}


\headertext{}

\addto{\captionsrussian}{\renewcommand{\bibname}{Список литературы}}

%\title{Исследование и реализация сферического коня в вакууме\\
  %на основе теоретико-множественного подхода}

\title{Сбор, анализ, разметка данных и создание среды обучения для виртуального актора на нейросетевой основе в контексте юмористических сюжетов}
  

\begin{document}

%\documenttype{Расширенное содержание пояснительной записки}
\documenttype{Пояснительная записка}

% Для титульного листа, сверстанного в LaTeX
\pagestyle{empty}
\input{title/blank-header}

\vfill

\begin{center}
  Направление подготовки 09.04.04 Программная инженерия

  \vfill

  {\Large{\textbf{\thedocumenttype}}}

  к научно-исследовательской работе студента на тему:

  {\Large\thetitle}
\end{center}

\vfill

{\large

\noindent
\begin{tabular}{@{}lcl@{}}
Группа              & $\rule{5cm}{0.15mm}$ & \theauthorgroup \\   
Студент             & $\rule{5cm}{0.15mm}$ & \theauthor \\    
Руководитель        & $\rule{5cm}{0.15mm}$ & \thesupervisor \\         
Научный консультант & $\rule{5cm}{0.15mm}$ & \theconsultant \\                
\end{tabular}

\vfill

\noindent
Оценка руководителя \quad $\rule{3cm}{0.15mm}$ \quad
Оценка комиссии     \quad $\rule{3cm}{0.15mm}$

\vfill

\noindent
\begin{tabular}{@{}lcc@{}}
Члены комиссии & & \\
& $\rule{6cm}{0.15mm}$ & $\rule{6cm}{0.15mm}$ \\
& $\rule{6cm}{0.15mm}$ & $\rule{6cm}{0.15mm}$ \\
& $\rule{6cm}{0.15mm}$ & $\rule{6cm}{0.15mm}$ \\
& $\rule{6cm}{0.15mm}$ & $\rule{6cm}{0.15mm}$ \\
\end{tabular}

\vfill

\begin{center}
\textbf{Москва \the\year}
\end{center}

}

\newpage

% Для вставки бланка титульного листа из PDF-файла
% \includepdf[pages={-}, offset=0mm -0mm]{title/title.pdf}

%\clearpage
%% Тут включается лист с подписями для ВКР
%\includepdf[pages={-}, offset=0mm -0mm]{title/title-dep22.pdf}
%\clearpage

% Для листа задания, сверстанного в LaTeX
\pagestyle{empty} 
\input{title/blank-header} 
 
\begin{center} 
  {\Large{\textbf{Задание на НИР}}} 
 
  \large 
  
  Студенту гр. \theauthorgroup{} \theauthorfulldative 
\end{center} 
 
\begin{center} 
  \uppercase{\textbf{\large{}Тема НИР}} 
 
  {\Large\thetitle} 
 
  \vskip 1em 
 
  \uppercase{\textbf{\large{}Задание}} 
\end{center} 
 
{\linespread{1.0} 
  \footnotesize 
  \noindent 
  \begin{longtable}{|p{.5cm}|p{250pt}|>{\centering\arraybackslash}p{2cm}|>{\centering\arraybackslash}p{2cm}|>{\centering\arraybackslash}p{2cm}|} \hline 
  \multicolumn{1}{|>{\centering\arraybackslash}p{0.5cm}|}{№\par п/п} & \multicolumn{1}{c|}{Содержание работы} & Форма отчетности & Срок исполнения & Отметка о выполнении\par {\scriptsize Дата, подпись} \\\hline 
\projecttask & Аналитическая часть &&& \\\hline 
% (указываются предмет и цели анализа) 
\projectsubtask & Изучение и анализ существующих когнитивныйх архитектур 
  & Пункт ПЗ 
  & 
  & 
  \\\hline 
\projectsubtask & Изучить методы машинного обучения 
  & Пункт ПЗ 
  & 
  & 
  \\\hline 
\projectsubtask & Изучение и анализ проблемной области распознования речи 
  & Пункт ПЗ 
  & 
  & 
  \\\hline 
\projectsubtask & Классификация и определение эмоций 
  & Пункт ПЗ 
  & 
  & 
  \\\hline 
\projectsubtask & Оформление расширенного содержания пояснительной записки (РСПЗ) & Текст РСПЗ & 10.04.2022 & \\\hline% 
\projecttask & Теоретическая часть &&& \\\hline 
\projectsubtask & Описать модель работы виртуального актора 
  & 
  & 
  & 
  \\\hline 
\projectsubtask & Разработка алгоритмов распознования речи 
  & Алгоритмы 
  & 
  & 
  \\\hline 
\projectsubtask & Модификация модификация методов машинного обучения применитьльно к задаче распознования речи 
  & Алгоритмы 
  & 
  & 
  \\\hline 
\projecttask & Инженерная часть &&& \\\hline 
\projectsubtask & Построить модель тембора голоса 
  & Модели 
  & 
  & 
  \\\hline 
\projectsubtask & Построить семантическую модель распознования голоса 
  & Модели 
  & 
  & 
  \\\hline 
\projectsubtask & Проектирование инструментов для анализа текста 
  & Макеты 
  & 
  & 
  \\\hline 
\projectsubtask & Проектирование приложения 
  & Макеты 
  & 
  & 
  \\\hline 
\projectsubtask & Результаты проектирования оформить с помощью UML диаграмм 
  & UML диаграммы 
  & 
  & 
  \\\hline 
\projecttask & Технологическая и практическая часть &&& \\\hline 
\projectsubtask & Реализация модели виртуального актора 
  & Исполняемые файлы, исходный текст  
  & 
  & \\\hline 
\projectsubtask & Реализация программного приложения 
  & Исполняемые файлы, исходный текст  
  & 
  & \\\hline 
\projectsubtask & Реализация и дообучение моделей машинного обучения 
  & Исполняемые файлы, исходный текст  
  & 
  & \\\hline 
\projecttask & Оформление пояснительной записки (ПЗ) и иллюстративного материала для доклада. & Текст ПЗ, презентация & 06.05.2022 & \\\hline 
\end{longtable} 
} 
%\refsection 
%\nocite{Sychev} 
%\nocite{Sokolov} 
%\nocite{Gaidaenko} 
%\begin{center} 
  %\uppercase{\textbf{\large{}Литература}} 
%\end{center} 
%\printbibliography[heading=none] 
%\endrefsection 
 
\begin{center} 
  \uppercase{\textbf{\large{}Литература}} 
\end{center} 
 
\begin{itemize} 
  \item Tikhomirova, D. V., Chubarov, A. A., Samsonovich, A. V. (2019). Empirical and
  modeling study of emotional state dynamics in social videogame paradigms.
  Cognitive Systems Research.

  \item Samsonovich A.V. Comparative Analysis of Implemented Cognitive Architectures//
  Biologically Inspired Cognitive Architectures. 2011. Vol. 233. P. 469-479. doi:
  10.3233/978-1-60750-959-2-469

  \item  Larue, O., West, R., Rosenbloom, P. S., Dancy, C. L., Samsonovich, A. V.,
  Petters, D., Juvina, I. (2018). Emotion in the common model of 74 A.V.
  Samsonovich / Cognitive
  Systems Research 60 (2020) 57–76 cognition. Procedia Computer Science, 145,
  740–746.

  \item  Madl, T., Franklin, S., Chen, K., Trappl, R. (2018). A computational cognitive
  framework of spatial memory in brains and robots. Cognitive Systems Research,
  47, 147– 172.

  \item  Laird, J. E., Lebiere, C., Rosenbloom, P. S. (2017). A standard model of the
  mind: Toward a common computational framework across artificial intelligence,
  cognitive science, neuroscience, and robotics. AI Magazine, 38(4), 13–26./

\end{itemize} 
 



\vfill 
 
{\noindent\linespread{2.0}

\begin{tabularx}{\linewidth}{p{140pt}XXX} 
  Дата выдачи задания: & Студент & \hrulefill & \theauthor \\ 
  10.10.2021           &   Руководитель    & \hrulefill & \thesupervisor \\ 
\end{tabularx} 
}

% Для вставки бланка задания из PDF-файла
%\includepdf[pages={-}, offset=0mm -0mm]{title/zadanie.pdf}
\newpage

\pagenumbering{arabic}

\refsection

%\clearpage
%\thispagestyle{empty}

%\vfill

%\begin{center}
%[Место для распечатки отчета Антиплагиата]
%\end{center}

%\newpage
%\thispagestyle{empty}

%\vfill

%\begin{center}
%[Место для распечатки отчета Антиплагиата]
%\end{center}

\clearpage

\chapter*{Реферат}
\thispagestyle{plain}


Пояснительная записка содержит - страниц, - рисунков, -  источников литературы.

Ключевые слова: UNITY3D, EBICA, СОЦИАЛЬНО-ЭМОЦИОНАЛЬНЫЙ ИНТЕЛЛЕКТ, ВИРТУАЛЬНЫЙ АКТОР, МАШИННОЕ ОБУЧЕНИЕ, РЕККУРЕНТНЫЕ СЕТИ 

Объектом исследования в данной работе является социально-эмоциональный интеллект

Предмет исследования в данной работе - модель когнитивной архитектуры для создания Виртуального Актора, 
а так же подходы машинного обучения расширяющие его функционал

В первом разделе обозначены когнитивные архитектуры, выявляются их преимущества и недостатки по 
сравнению с когнитивной архитектурой eBICA. Определяются типы распознавания и синтеза речи. 
Выделятся классификация эмоций.
Рассматриваются методы распознавания эмоций в тексте.
Ставятся цели и задачи научно-исследовательской работы. 

Во втором производится анализ когнитивных архитектур и производится анализ методов распознавания человеческой речи,
анализ подходов в машинном обучении для распознавания эмоций речи.
Так же приводятся теоретические выкладки описания модели социально-эмоционального интеллекта. 

В третьем разделе приводится подробное описание работы алгоритма, реализующего когнитивную модель социально-эмоционального интеллекта с учетом
интегрированной системы распознавания речи и распознавания эмоций в нейю
Строятся блок-схемы для кодовой реализации модели интеллекта.

В четвертом разделе приводятся реализация программного продукта представляющего собой когнитивно эмоциональный интеллект, 
интегрированную систему распознавания речи и выделения эмотивных признаков, показаны результаты в иллюстрациях.
а так же приведены способный взаимодействовать на эмоциональной основе.


\clearpage

\tableofcontents{}

\clearpage

\chapter*{Введение}
\label{sec:afterwords}
\addcontentsline{toc}{chapter}{Введение}


В последнее время все большую и большую популярность набирают технологии предоставляющие возможность участвовать человеку 
в виртуальном мире либо технические средства, позволяющие представление виртуальную реальность в реальном мире. 
Виртуальные Акторы способны в будущем заменить докладчика на конференциях различного характера и представлениях.

Целью исследования является создание общей вычислительной модели механизмов, лежащих в основе человеческих эмоций. Осуществляется это путем создания Виртуального агента, подкрепленного когнитивной архитектурой и помещенного в виртуальное окружение. В данной парадигме человек (испытуемый, проводящий сеанс игры) может взаимодействовать с Виртуальным Актором, воплощенного в виде аватара в игре с трехмерной графикой, и оценивать его по различным параметрам. Такая модель может интерпретировать человеческое поведение и на основе экспериментов можно делать выводы о ее социальной приемлемости и точности имитирования человеческих эмоций.

В первом разделе обозначены когнитивные архитектуры, выявляются их преимущества и недостатки по 
сравнению с когнитивной архитектурой eBICA. Определяются типы распознавания и синтеза речи. 
Ставятся цели и задачи научно-исследовательской работы. 

Во втором производится анализ когнитивных архитектур и производится анализ методов распознавания человеческой речи, а так же выявление в ней эмоциональных составляющих.
Так же приводятся теоретические выкладки описания модели социально-эмоционального интеллекта. 

В третьем разделе приводится подробное описание работы алгоритма, реализующего когнитивную модель социально-эмоционального интеллекта с учетом
интегрированной системы распознавания речи.
Строятся блок-схемы для кодовой реализации модели интеллекта.

В четвертом разделе приводятся реализация программного продукта представляющего собой когнитивно эмоциональный интеллект, 
способный взаимодействовать на эмоциональной основе.

\clearpage

\chapter{Исследование существующих когнитивных архитектур и анализ их недостатков}


Проводится анализ по выявлению существующих недоработок прототипа. 
Выявляются недостатки и преимущества по сравнению с другими моделями искусственного интеллекта.

%Это обзорно-аналитическая глава, в которой требуется отразить:

%\begin{itemize}
	%\item результат изучения различных существующих методов решения задач в рамках проблематики УИРа/диплома (иногда даже в смежных областях), это обзорный аспект, который пишется, в основном, на основе имеющейся литературы или/и программного обеспечения;
	%\item сравнение (с какой-либо определенной целью) этих методов и средств.
%\end{itemize}

%Приведенные ниже названия пунктов являются очень примерными, их состав и структура сильно зависят от специфики конкретной работы.




%Большие отсупы --- это хорошо. Облегчает чтение длинных <<простыней>> текста


%ок
\section{Изучение и анализ существующих когнитивных архитектур}

Одной из наиболее известных когнитивных архитектур является архитектура, 
составленная Jonathan Gratch и Stacy Marsella, что описано в работе \cite{Samsonovich03}.
Цель их исследования - создать общую вычислительную модель механизмов, 
лежащих в основе человеческих эмоций, которая сможет всецело их описать. 
Хотя такая модель может давать объяснение человеческого поведения, 
они рассматривают разработку вычислительных моделей эмоций как 
ключевой объект исследований для искусственного интеллекта, 
который будет способствовать развитию большого количества вычислительных систем,
которые моделируют, интерпретируют или влияют на человеческое поведение. 
На рисунке (Рис. \ref{pic:ris1}) демонстрируется Когнитивно-мотивационно-эмоциональная система.

\begin{figure}[h]
\includegraphics[width=0.75\columnwidth]{./img/ris1.png}
\centering
\caption{Когнитивно-мотивационно-эмоциональная система по материалам Smith and Lazarus.}
\label{pic:ris1}
\end{figure}

Теория оценки служит концептуальной основой их работы, но эта психологическая теория недостаточно точна, чтобы служить
спецификацией вычислительной модели. Для этого они переделывают теорию с точки зрения методов и представлений искусственного интеллекта. 
Когнитивно-мотивационно-эмоциональная система Craig Smith и Richard Lazarus, показанная на рисунке 1, является представителем современных
теорий оценки. Эмоция концептуализируется как двухступенчатая система контроля. Оценка характеризует отношения между человеком и его 
физическим и социальным окружением, называемые отношениями человека и окружающей среды, копирование поведения для восстановления или 
поддержания этих отношений. Поведение возникает в результате тесной связи познания, эмоций и реакций совладения: когнитивные процессы 
служат для построения индивидуальной интерпретации того, как внешние события соотносятся с его целями и желаниями 
(отношения человека и окружающей среды). Система использует эти характеристики для изменения отношений между человеком и окружающей средой,
мотивируя действия, которые изменяют среду (копирование, ориентированное на проблему), или мотивируя изменения в интерпретации этих 
отношений (копирование, ориентированное на эмоции).

Модель PAD была разработана Albert Mehrabian и James A. Russell в 1974 году для описания и измерения эмоциональных состояний, как говорится в Работе \cite{Samsonovich04}. 
В данной модели используются три числовых измерения для представления всех эмоций:
\begin{itemize}
	\item A — arousal (возбуждение);
	\item P — pleasure (удовольствие); 
	\item D — dominance (доминирование).
\end{itemize}

Модель PAD первоначально использовалась в теории психологии окружающей среды, а основное идеей модели было предположение о том,
что физическая среда влияет на людей через их эмоциональное воздействие. На основе данной модели были построены физиологическая 
теория эмоций и теория эмоциональных эпизодов. Также модель использовалась для изучения невербального общения, в потребительском 
маркетинге и при создании анимированных персонажей, которые выражают эмоции.

В модели PAD используются трехмерные шкалы, которые в теории могут иметь любые числовые значения:
\begin{itemize}
	\item шкала удовольствия-неудовольствия показывает, насколько приятно или, наоборот, неприятно человек себя чувствует по отношению к чему-то. Например, радость это — приятная эмоция; гнев и страх — неприятные эмоции;
	\item шкала возбуждения-неактивности измеряет, насколько человек чувствует возбуждение или его отсутствие. В данном случае оценивается именно возбуждение, а не интенсивность эмоций. Например, горе или депрессия характеризуются слабым возбуждением, но сильной интенсивностью; а гнев или ярость имеют и высокую интенсивность, и высокое состояние возбуждения;
	\item шкала доминирования-покорности описывает чувство контроля и доминирования по сравнению со смирением и подчиненностью. Например, гнев — это доминирующая эмоция, а страх
	\item эмоция покорности, хотя обе они имеют неприятный характер.
	\item эмоция покорности, хотя обе они имеют неприятный характер.
\end{itemize}

Еще одна интересная когнитивная архитектура описана в статье трех научных деятелей Ron Sun, Nick Wilson, 
Michael Lynch. Статья имеет название: “Emotion: A Unified Mechanistic Interpretation from a Cognitive Architecture”.
В этой статье рассматривается проект, который пытается интерпретировать эмоции - сложное и многогранное явление с 
механистической точки зрения, чему способствует существующая комплексная вычислительная когнитивная архитектура - CLARION.
Эта когнитивная архитектура состоит из ряда подсистем: подсистем, ориентированных на действие, не ориентированных надействия, 
мотивационной и метакогнитивной подсистем. С этой точки зрения эмоции в первую очередь основаны на мотивации.
Основываясь на этих функциональных возможностях, мы механистически (вычислительно) соединяем части вместе 
в рамках CLARION и фиксируем множество важных аспектов эмоций, как описано в литературе. 
На (Рис. \ref{pic:ris2}) демонстрируются подсистемы когнитивной архитектуры CLARION.
\begin{figure}[h]
\includegraphics[width=0.75\columnwidth]{./img/ris2.png}
\centering
\caption{Подсистемы когнитивной архитектуры CLARION}
\label{pic:ris2}
\end{figure}

Основные информационные потоки показаны стрелками. ACS означает подсистему, ориентированную на действия. NACS означает подсистему, не ориентированную на действия. МС — это мотивационная подсистема. MCS означает метакогнитивную подсистему.

Получившая наибольшее распространение из всех формальных моделей представления эмоций является модель OCC (Ortony, Clore, \& Collins), 
которая упоминается в работет \cite{Samsonovich06}, 
предложенная в 1988 году учеными Кембриджского университета. Иерархия содержит три ветви, а именно: эмоции, касающиеся последствий 
событий (например, радость и жалость), действия агентов (например, гордость и упрек) и аспекты объектов (например, любовь и ненависть). 
Кроме того, некоторые ветви объединяются в группу сложных эмоций, а именно эмоций относительно последствий событий, вызванных действиями 
агентов (например, благодарность и гнев). 
На рисунке (Рис. \ref{pic:ris3}) демонстрируется оригинальная модель OOC.

\begin{figure}[h]
\includegraphics[width=0.75\columnwidth]{./img/ris3.png}
\centering
\caption{Оригинальная модель OCC}
\label{pic:ris3}
\end{figure}
В основе правил динамики данной модели лежит реакция валентности (Valenced reaction). 
Под «валентностью» в психологии понимают внутреннюю привлекательность – «хорошую» 
(положительную валентность) или отвратительность – «плохую» (отрицательную валентность) 
события, объекта или ситуации. Эмоции формируются под воздействием трех основных факторов — 
последствий событий (Consequences of events), действий движущих сил (Actions of agents) и аспектов событий.

Рассмотрим левую «ветку» эмоциональной реакции. Последствия событий могут быть приносящими 
удовольствие (pleased) или доставляющими неудовольствие (displeased). Проведя предварительную 
оценку, человек фокусируется (Focusing on) на разделении последствий событий для себя 
(Consequences for self) и для других (Consequences for self), которые, в свою очередь могут 
оказаться для последних желательными (Desirable for other) или нежелательными (Undesirable for other). 
По поводу судеб других (Fortunes of others) в зависимости от личного отношения — положительного или 
отрицательного — человек может испытывать следующие эмоции: радость за другого (Happy for), 
обида (Resentment), злорадство (Gloating) или жалость (Pity).

У этой модели есть свои ограничения, заключающиеся как в ее требовании упрощения человеческих эмоций,
 так и в ее сложном подходе к тому, как надлежит выводить эмоциональные состояния конечных пользователей 
 посредством интерпретации поведения человека через знаки и сигналы, транслируемые людьми. Использование 
 этой модели в ее оригинальном описании затруднено отсутствием математического аппарата, в следствии чего 
 многие исследователи в своих Виртуальных Акторах используют упрощённые версии данной модели.

Также большой интерес представляет когнитивная архитектура, реализованная в физическом роботе, 
под названием - интегрированное когнитивное универсальное тело (iCub). Это когнитивная архитектура,  
дизайн которой  основан на  существующих знаниях в области робототехники, вычислений, нейробиологии и 
психологии, целью которой является копирование некоторых когнитивных процессов человека для их включения 
в человекоподобных роботов.

Эта архитектура реализована в человекоподобном роботе. Он был разработан для исследования сообществом когнитивных систем. 
Кроме того, он имеет лицензию «Стандартная общественная лицензия GNU (GPL)», так что любой человек может свободно использовать 
все наработки по данному проекту. Данная архитектура реализована в человекоподобном роботе, который имеет 53 степени свободы. 
По размеру он похож на ребенка трех-четырех лет и ребенка в возрасте 2,5 лет по когнитивным способностям. Кроме того, он может 
ползать и сидеть. Некоторые особенности, которые описаны в работе \cite{Samsonovich02}

\begin{itemize}
	\item	Не хватает семантической памяти, чтобы помочь ему обобщать события;
	\item Невозможно сформировать привычки;
	\item Он учится путем подражания, проб и ошибок;
	\item Обнаруживает, распознает и отслеживает человеческое лицо, наблюдая за его действиями; 
	\item Действия основаны на жестах рук, таких как встряхивание и манипулирование объектами,например, толкание, подъем и опускание. 
	\item Действия, наблюдаемые роботом, изучаются и сохраняются в базе данных в процессе обучения.
\end{itemize}

На рисунке (Рис. \ref{pic:ris4}) представлена схема работы iCub.
\begin{figure}[h]
\includegraphics[width=0.75\columnwidth]{./img/ris4.png}
\centering
\caption{Схема работы iCub}
\label{pic:ris4}
\end{figure}

Вспомогательным инструментом при создании актора, наделенного социально- эмоциональным интеллектом, 
может являться - имитация моторного обучения (IML). IML начинает наблюдать за другим актором, осуществляющим 
некоторую цепочку действий, затем категоризирует действия (определяет какую цель преследуют данные действия) 
одновременно отслеживая изменения точки обзора, окружающей среды, положения и типов объектов. Другими словами, 
когда Виртуальный агент неоднократно наблюдает за определенной новой последовательностью действий, каждый из 
знакомых элементов действия активирует соответствующее моторное представление через существующие ассоциации. 
Данное наблюдение формирует связи между элементарными моторными представлениями. Эта связь представлений 
составляет моторное обучение и улучшает имитационное движение. Способность моторной системы интегрировать 
разные части организма позволила бы создать обширный репертуар моторного поведения путем смешивания выходных 
сигналов разных частей организма, чтобы конечный результат отражал относительный и взвешенный вклад каждого в 
достижении цельной имитации движения. Поскольку невозможно воспроизвести функционирование мозга, были созданы 
модели, которые пытаются имитировать различные функции и поведение.
%ок
\section{Изучение и анализ когнитивной архитектуры eBICA}

Архитектура состоит из семи компонентов: интерфейсный буфер, рабочая, процедурная, семантическая 
и эпизодическая системы памяти, система ценностей и система когнитивных карт. Три основных строительных 
блока для этих компонентов — это ментальные состояния, схемы и семантические карты. 
Семантическая память — это коллекция определений схем. Буфер интерфейса заполняется схемами. 
Рабочая память включает активные психические состояния. Эпизодическая память хранит неактивные психические состояния,
сгруппированные в эпизоды - предыдущее содержимое рабочей памяти. Следовательно, эпизодическая память состоит из структур, 
аналогичных тем, которые обнаруживаются в рабочей памяти, но которые «заморожены» в долговременной памяти \cite{seman_karta}. Процедурная 
память включает в себя примитивы. Система ценностей включает в себя шкалы, представляющие основные значения. 
Система когнитивных карт включает, в частности, семантические карты эмоциональных ценностей. 
Семантическая карта использует абстрактное метрическое пространство (семантическое пространство) для представления
семантических отношений между ментальными состояниями, схемами и их 13 экземплярами, а также для присвоения значений их оценкам. 
На (Рис.\ref{pic:ris5}) демонстрируется семантическая карта \cite{seman_karta}.

\begin{figure}[h]
\includegraphics[width=0.75\columnwidth]{./img/ris5.png}
\centering
\caption{Семантическая карта}
\label{pic:ris5}
\end{figure}

Для когнитивного семантического отображения может использоваться слабое когнитивное семантическое картирование. 
Идея заключается в том, чтобы расположить представления на основе очень немногих основных семантических измерениях. 
Эти измерения могут возникать автоматически, если стратегия состоит в том, чтобы объединить синонимы и антонимы друг от друга. 
Карта, часть которой показана на рисунке 6 является результатом этого процесса. Эта карта не очень хорошо отделяет различные 
значения друг от друга: например, основные и сложные чувства. Однако она классифицирует значения в соответствии с их семантикой. 
Рисунок (Рис. \ref{pic:ris6}) демонстрирует примеры простейших эмоциональных элементов в рамках eBICA \cite{Samsonovich01}.

\begin{figure}[h]
\includegraphics[width=0.75\columnwidth]{./img/ris6.png}
\centering
\caption{примеры эмоциональных элементов в рамках eBICA}
\label{pic:ris6}
\end{figure}

(A) Схема имеет оценку в качестве своего атрибута. Это также атрибут головного узла. Значение этого атрибута - «доминантный», 
что означает, что действие воспринимается как проявление доминирования, или агент воспринимается как «доминантный по отношению ко мне» 
и т. Д. (B) Психическое состояние имеет оценку атрибут, который представляет собой эмоциональное состояние и самооценку агента в данный 
момент в данной ментальной перспективе. Показанная ценность этой оценки «взволнована», что означает, что агент находится в возбужденном 
эмоциональном состоянии. Моральная схема, показанная в B, связывается с частью содержания психического состояния (включая определенный 
образец оценок) и представляет оценку выбранного образца, например, образец взаимодействий и взаимных оценок двух агентов, упомянутых в 
ментальном состоянии.

%можно еще дописать
\section{Нейронные сети и их типы}

Нейро́нная сеть (также искусственная нейронная сеть, ИНС) — математическая модель, а также её программное или аппаратное воплощение, 
построенная по принципу организации и функционирования биологических нейронных сетей — сетей нервных клеток живого организма. 
Это понятие возникло при изучении процессов, протекающих в мозге, и при попытке смоделировать эти процессы. Первой такой попыткой 
были нейронные сети У. Маккалока и У. Питтса. После разработки алгоритмов обучения получаемые модели стали использовать в практических 
целях: в задачах прогнозирования, для распознавания образов, в задачах управления и др. 

Разделяют несколько основных разновидностей Нейронных сетей, согласно работе \cite{neural01}, а именно:
\begin{itemize}
	\item Нейронные сети прямого распространения
	\item Сети радиально-базисных функций
	\item Нейронная сеть Хопфилда (Hopfield network, HN)
	\item Цепи Маркова (Markov chains, MC или discrete time Markov Chains, DTMC)
	\item Машина Больцмана (Boltzmann machine, BM)
	\item Ограниченная машина Больцмана (restricted Boltzmann machine, RBM)
	\item Автокодировщик (autoencoder, AE)
	\item Разреженный автокодировщик (sparse autoencoder, SAE)
	\item Вариационные автокодировщики (variational autoencoder, VAE)
	\item Шумоподавляющие автокодировщики (denoising autoencoder, DAE)
	\item Сеть типа «deep belief» (deep belief networks, DBN)
	\item Свёрточные нейронные сети (convolutional neural networks, CNN)
	\item Развёртывающие нейронные сети (deconvolutional networks, DN)
\end{itemize}

С точки зрения машинного обучения, нейронная сеть представляет собой частный случай методов распознавания образов, дискриминантного анализа.
Рекуррентные нейронные сети (РНС, англ. Recurrent neural network; RNN) — вид нейронных сетей, где связи между элементами образуют 
направленную последовательность \cite{Wikipedia01}. Благодаря этому появляется возможность обрабатывать серии событий во времени или последовательные 
пространственные цепочки. В отличие от многослойных перцептронов, рекуррентные сети могут использовать свою внутреннюю память для 
обработки последовательностей произвольной длины. Поэтому сети RNN применимы в таких задачах, где нечто целостное разбито на части, 
например: распознавание рукописного текста или распознавание речи. Было предложено много различных архитектурных решений для 
рекуррентных сетей от простых до сложных. В последнее время наибольшее распространение получили сеть с долговременной и 
кратковременной памятью (LSTM) и управляемый рекуррентный блок (GRU).

В последнее время наибольшую популярность для решения задач тематической классификации, применяемой для выделения семантического смысла текста,
приобрели глубокие нейронные сети, так как они позволяют достичь наивысшей точности среди всех известных моделей машинного обучения. 
В частности, сверточные нейронные сети совершили прорыв в классификации изображений. В настоящее время они успешно справляются и с 
некоторыми задачами автоматической обработки текстов. Более того, как утверждается в некоторых исследованиях сверточные сети подходят 
для этого даже лучше рекуррентных нейронных сетей, которые чаще всего используются для анализа текстовых последовательностей. 
С другой стороны, использование сверточных сетей для классификации текстов мало исследовано. Поэтому исследование применения 
сверточных нейронных сетей для задачи классификации текстов в качестве альтернативы рекуррентным нейронным сетям представляет 
практический интерес, что описано в \cite{neural10}.

Для решения поставленной задачи требуется получить способ представления данных в виде, пригодном для обработки сверточной нейронной сетью. 
Например, в виде матрицы вещественных чисел. Наиболее распространенным является способ отображения каждого слова в многомерное 
векторное пространство. В рамках данной работы векторные представления слов строились на основе модели word2vec \cite{cyberlinka01}.
sa

\section{Методы определения эмоций в речи}

Для распозавнания эмоций в речи используют следующие подходы:
\begin{itemize}
	\item Анализ эмотивной сотсавляющей текста в речи
	\item Анализ тональности речи
\end{itemize}


На текущий момент сущестуют следующие методы определения тональности текста:
\begin{enumerate}
  \item Анализ текста методами векторного анализа (часто с применением n-граммных моделей), сравнение с ранее размеченным эталонным корпусом по выбранной мере близости и отнесение (классификация) текста к негативу или позитиву на основании полученного результата сравнения.
  \item Поиск эмотивной лексики (лексической тональности) в тексте по заранее составленным тональным словарям (спискам паттернов) с применением лингвистического анализа. По совокупности найденной эмотивной лексики текст может быть оценен по шкале, отражающей количество негативной и позитивной лексики. Этот метод может использовать как списки паттернов, подставляемые в регулярные выражения, так и правила соединения тональной лексики внутри предложения
  \item Смешанный метод (комбинация первого и второго подходов).
\end{enumerate}

% тут надо нормально проставить cit
Первый метод (см., например, [Pang \& al., 2002; Pang \& al., 2005; Gamon,
2004]) работает достаточно быстро, но требует наличия предварительно размеченного эталонного корпуса, на основе которого происходит обучение алгоритма сравнения. 
Существенными недостатками такого подхода оказываются увеличение трудоемкости и ограничение разнородности корпуса (т. е.
неполнота лексического покрытия), что приводит к потере точности. К тому же
данный метод не позволяет провести глубокий анализ текста, то есть выявить
и показать эмотивность на уровне предложения.
% тут надо нормально проставить cit

Второй метод [Nasukawa, 2003; Yi, 2003] не менее трудоемок в составлении тональных словарей (или получения списка тональных паттернов),
Метод определения эмоций в текстах на русском языке 513
но в сочетании с синтаксическим и морфологическим анализом более гибок:
он позволяет не только показать цепочки тональной лексики, но и получить
синтаксически корректные эмоциональные выражения. При хорошем наполнении тональных словарных списков этот метод позволяет достичь хорошей
полноты (покрытия эмотивной лексики).
% тут надо нормально проставить cit
Недостаток этого метода в том, что с помощью него сложно дать количественную оценку негативности-позитивности текста. 
Чтобы избежать недостатков первого и второго метода, используют смешанный подход [Prabowo \&
al., 2009; Konig, 2006], частично включающий в себя два первых.

Следующий метод определения эмоций в речи это их распознавание, посредством акустического анализа.

Индивидуальность голоса обеспечивается сочетанием поведенческих и
физиологических признаков. К поведенческим относят семантику, дикцию,
произношение, ритм, интонации и др. Они обусловлены социальными факто-
рами и могут быть довольно изменчивыми в зависимости от ситуации. Более
надежными являются анатомические особенности речевого тракта, поэтому
для работы автоматического распознавания наиболее адаптированы алгорит-
мы измерения акустических характеристик.

Акустическая теория речи рассматривает речевую волну как результат ра-
боты источника звука и фильтров. Подробное изложение о физиологических
процессах речеобразования и моделях речевого тракта можно найти в кни-
гах [2–4]. Здесь же кратко приведены только те параметры, которые участву-
ют в автоматическом распознавании дикторов.
Характерные черты голоса конкретного человека в цифровой обработке
сигналов получают через спектральный анализ речевой волны.

Частота первой гармоники спектра является частотой основного тона
(основной частотой голоса). Частота основного тона F0 – обратная величина
длительности T0 одного цикла работы голосовых связок: F0 = 1/ T0 . Основная
частота определяет высоту голоса – ощущение, связанное с воздействием тона
на слуховую систему человека.
Индивидуальность данного параметра объясняется тем, что длительность
T0 зависит от массы и упругости голосовых связок, а также от перепада дав-
ления над и под связками. Поэтому пол и возраст диктора оказывают влияние
на значения основной частоты.
Каждый человек имеет свой диапазон изменений частоты основного тона.
Как правило, для взрослого он составляет от полутора до двух октав. В задаче
распознавания личности по голосу необходимо определять базовую основную
частоту, т. е. привычный и удобный для идентифицируемого человека режим
работы голосовых связок.

\section{Классификации и определение эмоций}

Многообразие эмоций, их качественных и количественных проявлений исключают возможность простой и единой классификации. 
Каждая из характеристик эмоций может выступать в качестве самостоятельного критерия, основания для их классификации (таб. \ref{tbl:text_a00}).

\begin{table}[H]
\caption{характеристики эмоции как основания для их классификации}
\label{tbl:text_a00}
\begin{center}
%\centering

\begin{tabular}{ | c | c | }
	\hline
	Знак & Пололожительные, отрицательные, амбивалентные \\ \hline 
	Модальность & Радость, гнев, страх и др. \\ \hline
	Влияение на поведение и деятельность & Осознаваемые, неосознаваемые \\ \hline
	Предметность	& Предметные, беспредментные \\ \hline
	Степень произвольности & Произвольные, непроизвоольные \\ \hline
	Происхождение и развлечения & Врожденные, приобретенные, первичные, вторичные \\ \hline
	Уровень & Высшие, низшие \\ \hline
	Длительность & Кратковременные, длительные \\ \hline
	Интенсивность & Слабые, сильные \\ \hline
\end{tabular}
\end{center}
\end{table}

По знаку эмоциональные переживания можно разделить:
\begin{enumerate}
	\item на положительные
	\item отрицательные
	\item амбивалентные
\end{enumerate}

Основной функцией положительных эмоций является поддержание контакта с позитивным событием, поэтому им присуща реакция приближения к полезному, 
необходимому стимулу. Кроме того, по мнению П.В. Симонова, они побуждают нарушать достигнутое равновесие с окружающей средой и искать новую стимуляцию.

Для отрицательных эмоций характерной является реакция удаления, прерывания контакта с вредным или опасным стимулом.
Считается, что они играют более важную биологическую роль, поскольку обеспечивают выживание индивида.

Амбивалентными эмоциями являются противоречивые эмоциональные переживания, связанные с двойственным отношением к чему-либо или кому-либо (одновременное принятие и отвержение).

\section{Выводы}

Были изучены и проанализированы основные когнитивные архитектуры, особое внимание уделялось когнитивной архитектуры eBICA.
Была рассмотрена проблема синтеза и распознавание речи. Были изучены материалы описывающие классификацию и определение эмоций.



\clearpage

\chapter{Описание моделей, отвечающих за генерацию поведения виртуального актора}


В данном разделе приводится теоретическое описание модели.




\section{Постановка задачи}

В рамках научно-исследовательской работы был выявлен альтернативный путь решения поставленной задачи задачи, который выражается 
в обучения нейронной сети. Данный метод рассматривается параллельно реализации с использованием когнитивной архитектурой eBICA. 
eBICA – “emotional biologically inspired cognitive architecture” – “эмоциональная биологически вдохновленная когнитивная архитектура”. 
В этой архитектуре эмоциональные элементы добавлены практически ко всем процессам за счет модификации основных строительных блоков архитектуры. 
Ключевым моментом этой когнитивной архитектуры являются оценки, которые связаны со схемами и психическими состояниями как их атрибуты, 
моральные схемы, которые контролируют модели оценок и представляют социальные эмоции, а также семантические пространства, которые дают 
значения этих оценок.

Как видно из (Рис. \ref{pic:ris7}), архитектура представляет собой конгломерат компонентов: интерфейсный буфер, рабочая, процедурная, семантическая 
и эпизодическая системы памяти, система ценностей и система когнитивных карт \cite{Samsonovich01}. Три основных строительных блока для этих компонентов - это 
ментальные состояния, схемы и семантические карты. Семантическая память - это коллекция определений схем. Буфер интерфейса заполняется схемами.

\begin{figure}[h]
\includegraphics[width=0.75\columnwidth]{./img/ris7.png}
\centering  
\caption{Структура когнитивной архитектуры eBICA}
\label{pic:ris7}
\end{figure}

Рабочая память включает активные психические состояния. Эпизодическая память хранит неактивные психические состояния, сгруппированные 
в эпизоды - предыдущее содержимое рабочей памяти. Следовательно, эпизодическая память состоит из структур, аналогичных тем, которые 
обнаруживаются в рабочей памяти, но которые «заморожены» в долговременной памяти. Процедурная память включает в себя примитивы. Система 
ценностей включает в себя шкалы, представляющие основные значения. Система когнитивных карт включает, в частности, семантические карты 
эмоциональных ценностей. Семантическая карта использует абстрактное метрическое пространство (семантическое пространство) для представления 
семантических отношений между ментальными состояниями, схемами и их экземплярами, а также для присвоения значений их оценкам.

\section{Сбор данных о юморе}
Для обучения нейронной осуществляется сбор данных. Самой частой альтернативой сбора данных является его получение из открытых ресурсов при помощи парсинга кода HTML.

Для парсинга страниц используются несколько основных методов описанных в \cite{parser02}.
\begin{enumerate}
  \item Парсинг получаемого JSON при помощи Api
  \item Парсинг по XHR запросам в консоли разработчика браузера
  \item Поиск JSON в html странице
  \item Рендеринг код страницы через автоматизацию браузера
\end{enumerate}

В частности, используется нейросетевой подход для парсинга данных. Такой подход актуален так как владельцы информационных продуктов 
часто не заинтересованы в том, чтобы пользователи парсили данные с их ресурсов и добавляют различные ограничения и алгоритмы по распознаванию 
парсинга. Самым простым примером является ограничение на количество запросов к сайту, вставка капчи, проверка IP пользователя на соответствие 
региону либо количество посещений страницы в минуту. Так же очень частой проблемой является динамическая доработка сайтов, которая в следствии 
заставляет переделывать алгоритм парсинга.

Для того, чтобы парсинг страниц был более практичным, используют нейросетевой подход. Одним из таких инструментов является TensorFlow, 
это открытая программная библиотека для машинного обучения, разработанная компанией Google для решения задач построения и тренировки 
нейронной сети с целью автоматического нахождения и классификации образов, достигая качества человеческого восприятия. 

Для решения проблемы с IP используют спуфинг, что представляет из себя принцип маскировки одного юнита под другого путём фальсификации 
данных и позволяет получить соответствующие преимущества. 

Для анализа были взяты данные с различных сайтов с юмористическим посылом. В первую очередь рассматривались сайты со скетчами или сценками 
(в том числе скетчи из различных юмористических передач). Для каждого скетча и сценки на сайте должно быть их текстовое описание, поясняющее 
сюжет или сценарий. Далее рассматривались сайты с анекдотами и различными шутками. Такой приоритет был поставлен в связи с тем, что основой 
юмора в клоунаде должны быть действия акторов на сцене, а не юмористический эффект, полученный в результате “игры слов”. 

\section{Предварительная обработка полученных данных}

На вход модели должен подаваться заранее обработанный корпус текстов. Предварительная обработка состоит из следующих этапов \cite{seman06}:

\begin{itemize}
  \item Удаление всех знаков препинания, чисел и слов «нецелевых» языков (не предназначенных для обработки моделью).
  \item Разбиение текста на предложения. Для этого был выбран пакет библиотек Natural Language Toolkit (NLTK). Данная библиотека применяет регулярные выражения, а также некоторые алгоритмы машинного обучения для обработки естественного языка. Базовая версия NLTK не поддерживает разбиение русскоязычных текстов на предложения, поэтому использовалась модификация, расширяющая функционал библиотеки.
  \item Удаление «стоп-слов» – слов, не несущих определенной смысловой нагрузки, но при этом затрудняющих обработку исходного текста. Обычно для каждой специфической задачи применяется свой словарь стоп-слов, однако для нашей задачи достаточно стандартного словаря, содержащего буквы, частицы, предлоги, союзы, местоимения, числительные. Установлено, что удаление стоп-слов из тренировочного набора значительно снижает вычислительную стоимость, а также повышает точность модели.
  \item К корпусу текстов применяется стемминг или лемматизация. Это позволяет сократить размер словаря и искать семантически близкие слова, а не разные формы одного слова. Стемминг – это поиск основы слова, причем не обязательно совпадающей с корнем. Он имеет высокую скорость работы, но наиболее эффективен для английского языка, так как в нем для нахождения основы слова обычно достаточно удалить окончание. Для русского языка стемминг малоэффективен, поэтому применяется более ресурсоемкий алгоритм лемматизации. Лемматизация – это процесс приведения слова к начальной форме. В данной работе лемматизация осуществлялась морфологическим анализатором MyStem .
  \item Дополнение предложений до одинаковой длины с использованием нейтрального слова, так как сверточные нейронные сети способны обрабатывать только последовательности одинаковой длины.
\end{itemize}

Далее на начальном этапе необходимо перевести слова естественного языка в форму, пригодную для анализа сверточной нейронной сетью. 
Для этого лучше всего подходит векторное представление слов. Кроме того, среди всех моделей выберем ту, которая наиболее точно отражает 
реальные взаимосвязи между словами, а именно семантическую близость. Отметим, что модель не должна быть слишком требовательной к вычислительным 
ресурсам, чтобы было возможно совершать обучение сети на достаточно больших объемах данных.

Для выявления семантических связей между словами воспользуемся предположением лингвистики – 
дистрибутивной гипотезой: лингвистические единицы, встречающиеся в схожих контекстах, имеют близкие значения. 
Во многих моделях обработки текстов входные данные кодируются унарным кодом (one-hotencoding) – вектором, размерность которого равна мощности словаря более подробно 
описанный в \cite{neural02}. Элемент, соответствующий И.А. Батраева, А.Д. Нарцев, А.С. Лезгян номеру слова в словаре, равен единице, а остальные элементы равны нулю.
Однако у этого метода, согласно \cite{neural02} есть ряд существенных недостатков:

\begin{itemize}
  \item словари естественных языков могут быть достаточно объемными и исчисляться десятками и сотнями тысяч слов; следовательно, если каждое слово кодировать таким вектором, объем данных становится слишком большим;
  \item при таком способе кодирования теряется связь между словами: все слова считаются разными и никак не связанными между собой.
\end{itemize}

В силу вышесказанного, one-hot encoding не подходит для анализа семантической близостислов. Поэтому для данной задачи воспользуемся другим способом кодирования – распределенным представлением слов.
Распределенное (или векторное) представление слов – это способ представления слов в виде векторов евклидова пространства, размерность которого обычно равна нескольким сотням. Основная идея заключается в том, что геометрические отношения между точками евклидова пространства будут соответствовать семантическим отношениям между словами. Например, слова, представленные двумя близко расположенными точками векторного пространства, будут, скорее всего, синонимами или просто тесно связанными по смыслу словами. Семантическая близость слов вычисляется как расстояние между векторами, для чего используется так называемая косинусная мера.

Word2vec включает в себя две различные архитектуры – CBOW (Continuous Bag of Words – непрерывный мешок слов) и Skip-gram \cite{seman01}. CBOW пытается предсказать слово, исходя из текущего контекста, а Skip-gram, наоборот, пытается предсказать контекст по текущему слову. Для реализации модели была выбрана архитектура Skip-gram, которая, несмотря на меньшую скорость обучения, лучше работает с редкими словами.
Предварительно обработанный текст можно подавать на вход модели, после чего будут выполнены следующие действия:

\begin{itemize}
  \item считывается корпус текстов и рассчитывается, сколько раз в нем встретилось каждое слово; 
  \item из этих слов формируется словарь, который сортируется по частоте слов; также из словаря
  \item для сокращения его размера удаляются редкие слова;
  \item модель идет по субпредложению (обычно предложение исходного текста или абзац) окном определенного размера; под размером окна понимается максимальная длина между текущим словом и словом, которое предсказывается. Оптимальный размер окна – 10 слов; 
  \item	к данным, находящимся в текущем окне, применяется нейронная сеть прямого распространения с линейной функцией активации скрытого слоя и функцией активации softmax для выходного слоя.
\end{itemize}

Из всего вышесказанного ясно, что матрицы, задающие скрытый и выходной слои, получаются чрезвычайно большими. 
Это делает обучение сети долгим процессом. Поэтому используются различные оптимизации, которые позволяют существенно 
снизить, согласно \cite{neural03} временные и вычислительные затраты, незначительно потеряв в точности. Одной из таких модификаций является субсемплирование.
Чтобы избежать дисбаланса между редкими и часто встречающимися словами, используется простой подход: каждое слово 
отбрасывается с вероятностью, зависящей от частоты вхождения этого слова в текст.

\section{Описание работы модели актора в старом приложении}

На момент начала выполнения работы уже была реализована система работы виртуальных агентов, и она состоит в том, что сперва 
считываются действия и объекты с заданными для них параметрами и значениями из Excel файла, а также инициализируются значения. 
Затем выбор действий происходит в следующем порядке: в радиусе вокруг оценок виртуального актора выбираются действия, которые 
попадают в этот радиус, а также проверяются различные условия, необходимые для выполнения действия. Если условия не выполняются, 
то действие не может быть выбрано. Затем, после того как в список добавлены все действия, которые могут быть выполнены, рассчитываются 
вероятности на основе оценок и рассчитанных констант. Также, если действие повторное, его вероятность несколько занижается. 
После расчета вероятностей выбирается действие, которое влияет на оценки самого клоуна, а также, если есть, на цель клоуна. 

Помимо этого происходит замена состояний объектов и виртуальных акторов. После этого происходит перерасчет оценок Appraisals и Feelings \cite{Samsonovich01}.

Основная модель eBICA определяет поведение виртуального актора исходя из следующих факторов:

\begin{itemize}
  \item соматический;
  \item рациональный;
  \item когнитивный.
\end{itemize}

Нравственный фактор регулирует отношения первого актора со вторым на основе системы ценностей (представленной семантической картой) 
и моральных схем. Под когнитивным фактором понимается учет соображений нравственности, этики и морали, общей системы ценностей, 
понятий о добре и зле, о собственном достоинстве, эмпатии, соображений эстетики, стремлений к простоте и элегантности, и т.д. 
Учет этих соображений возможен на основе когнитивных оценок (appraisals) всех релевантных агентов, событий, их возможных действий 
и последствий этих действий, фактов, свойств, отношений, и т.д. Возможен вариант модели, в которой ответное действие может выбираться 
лишь из двух вариантов: положительная реакция на действие человека и отрицательная. Данная версия модели весьма неплохо работает даже 
с таким ограничением. Но невозможно придерживаться данной парадигмы при увеличении количества возможных вариантов для взаимодействия 
между акторами. В данной модели необходимо учесть пересчет оценок Appraisals и Feelings. Для пересчета оценок Appraisals используется 
следующая формула \ref{eq:appraisals01}:

\begin{equation}
  \begin{gathered}
    Appraisals=(1-r)*Appraisals+r*Action
    %P_{i+1j  } (R_{i+1j  } , {\varphi}_{i+1} , {\theta}_{j  }) \\
  \end{gathered}
  \label{eq:appraisals01}
\end{equation}

где Appraisals - оценка, 
r - эмпирически вычисленная константа экспоненциального затухания, 
Action - оценка совершаемого действия на семантической карте.

Одновременно с Appraisals пересчитываются так называемые “чувства” Feelings согласно режиму работы моральной схемы.
Аффективное пространство VAD – это трехмерное векторное пространство, точки которого соответствуют определенным эмоциональным 
состояниям, или аффектам, представленным триплетами значений (Valence, Arousal, Dominance). 
Существуют и сходные модели: PAD (Pleasure, Arousal, Dominance), EPA (Evaluation, Potency, Arousal) и другие. 
Здесь мы используем модель VAD. Соответственно, под «семантической картой» здесь часто понимается ее конкретная 
разновидность: аффективная карта (или когнитивная семантическая карта).

Шкалы имеют следующие значения:
\begin{itemize}
  \item dominance – варьируется при значении от 0 (покорность) до +1 (доминантность) и описывает соответствующие чувства; 
  \item valense – при значениях от -1 до 0 показывает уровень негатива или радости соответственно; 
  \item arousal – значения от -1 до 1 показывают уровень возбуждения (заинтересованности), к примеру, гнев по уровню возбуждения сильнее раздражительности, но слабее ярости. 
\end{itemize}

Оценки представлены в виде векторов на трехмерной семантической карте \cite{seman_karta}, \ref{pic:ris5}.
Моральная схема определяет общую установку на оценку поведения акторов, согласно их ролям и типу ситуации. 
Ее целью (как агента) является достижение и поддержание «нормального» положения дел, определенного набором Feelings. 
Вообще говоря, моральная схема состоит из двух частей: части, распознающей тип ситуации и осуществляющей привязку (binding),
и части, реализующей динамику схемы. В случае парадигмы актора можно считать, что моральная схема одна, уже привязана, и 
потому первая часть ее не актуальна.

Субъективные оценки (Feelings) генерируются по определенным правилам на основании истории объективных оценок и состояний системы. 
Грубо говоря, Feelings – это субъективное представление о том, каким оцениваемый актор является «на самом деле», и, следовательно, 
какого поведения от него нужно ожидать и на какое место его нужно ставить своим поведением. Следовательно, выбор поведения актора 
должен осуществляться так, чтобы приблизить Appraisals к Feelings. 

Значение Feelings определяет моральная схема, которая может работать в одном из трех режимов. 
Первый режим основывается на формуле \ref{eq:feelings01}:

\begin{equation}
  \begin{gathered}
    Feelings=beta*Appraisals
  \end{gathered}
  \label{eq:feelings01}
\end{equation}

где beta – эмпирически вычисленная константа. 

В данном режиме схема говорит, что если актор ведет себя хорошо, то к нему нужно относиться как к хорошему, и т.д. 

Цель данного процесса – распознать и классифицировать актора, выработать отношение к нему и приписать ему определенную роль во взаимоотношениях. 

В данном режиме моральная схема работает пока разница между квадратами норм Feeling и Appraisals не станет меньше некоторого значения.

Суть второго режима заключается в том, что значение Feeling фиксировано и экстремально по абсолютной величине, т.е. находится на сфере, 
ограничивающей семантическую карту (предположим, что есть такая сфера). Направленность вектора Feeling может быть либо произвольной, 
определенной предысторией, либо дискретной – вдоль одной из осей.

Третий режим состоит в том, что значения Feelings меняются \ref{eq:feelings02}, подстраиваясь под текущие значения Appraisals (здесь r1 может быть отличным от r): 

\begin{equation}
  \begin{gathered}
    Feelings=(1-r_1 )*Feelings+r1*(Appraisal-Feelings)
  \end{gathered}
  \label{eq:feelings02}
\end{equation}

Соответственно значения Appraisals и Feelings как говорится в работе \cite{Samsonovich05} пересчитываются после каждого действия первого актора, направленного на второго актора.
Также пересчет оценок происходит после определения и совершения одним из акторов ответного или самостоятельного действия. 
В данном контексте под термином “самостоятельное действие” имеется в виду действие, основанное лишь на текущем состоянии мира и 
значений векторов Appraisals и Feelings акторов, отобрежнные на (Рис. \ref{pic:ris8}). 


\begin{figure}[h]
\includegraphics[width=0.75\columnwidth]{./img/ris8.png}
\centering
\caption{Корреляция значений Appraisals (оранжевая) и Feelings (синяя) для показателя доминантности на протяжении времени/действий с шагом в 5 секунд}
\label{pic:ris8}
\end{figure}

Согласно (Рис. \ref{pic:ris8}) мы видим, что работа модели сводится к выбору действия, 
которое будет максимально приближать Appraisals к Feelings и вектор соматического состояния к начальному положению.

\section{Рекуррентные нейронные сети}

LSTM — это класс возвратных нейронных сетей. Поэтому, прежде чем мы сможем перейти к LSTM, 
важно понять нейронные сети и рекуррентные нейронные сети \cite{Wikipedia01}. 

Нейронные сети - Искусственная нейронная сеть представляет собой слоистую структуру из связанных нейронов,
вдохновленную биологическими нейронными сетями. Это не один алгоритм, а комбинация различных алгоритмов, 
которая позволяет нам выполнять сложные операции с данными. 
Рекуррентные нейронные сети - это класс нейронных сетей, предназначенных для работы с временными данными. 
Нейроны RNN имеют состояние / память ячейки, и ввод обрабатывается в соответствии 
с этим внутренним состоянием, которое достигается с помощью петель в нейронной сети. 
В RNN существуют повторяющиеся модули «tanh» слоев, которые позволяют им сохранять информацию. 
Однако ненадолго, поэтому нам нужны модели LSTM. 

LSTM - Это особый вид рекуррентной нейронной сети, способной изучать долгосрочные зависимости в данных. 
Это достигается за счет того, что повторяющийся модуль модели имеет комбинацию четырех слоев, взаимодействующих друг с другом \cite{Wikipedia01}. 

На (Рис. \ref{pic:ris9}) отображены структуры вышеупомянутые виды рекуррентных сетей: 

\begin{figure}[h]
\includegraphics[width=0.75\columnwidth]{./img/ris9.png}
\centering
\caption{Структуры рекуррентных нейронных сетей}
\label{pic:ris9}
\end{figure}

\section{Выводы}

В данном разделе была сформулирована постановка задачи, рассмотрены возможные ее решения с помощью архитектуры eBICA и нейронных сетей. 
Обозначена проблема сбора данных для обучения нейронной сети с веб-сервисов, и были выявлены способы решения данной проблемы. 
А также рассмотрена проблема обработки данных применительно к модели действий виртуальных агентов. Обозначены основные положения 
поведения виртуальных агентов. Выбран вид нейронной сети, которая будет моделировать поведение виртуальных акторов, а именно 
рекуррентная нейронная сеть.


\clearpage

\chapter{Проектирование модели поведения виртуального агента}

В этом разделе описывается и обосновывается выбор инструментария для проектирования и программного воплощения 
модели поведения актора в заданной парадигме. Описываются ключевые моменты проектирования и программной реализации модели поведения актора.

\section{Описание предыдущей модели поведения актора и виртуального окружения}

В ходе выполнения программы записывались данные с оценками для первого и второго клоуна, а также с выводом сообщения совершенного 
действия, данные представлены на (Рис. \ref{pic:ris10}).

\begin{figure}[h]
\includegraphics[width=0.75\columnwidth]{./img/ris10.png}
\centering
\caption{Оценки и действия совершенные первым виртуальным актором}
\label{pic:ris10}
\end{figure}

Для осуществления контроля за действиями акторов и их анализом система ведет записи в журнал событий. 
Журнал представляет собой текстовую таблицу. Записи выглядят следующим образом: каждая строка журнала соответствует своему событию, 
строка начинается с указания времени, когда была произведена запись. После указания времени указывается тип сообщения. 
Далее в сообщении показывается исполнитель действия, цель действия и номер действия из таблицы действий. 
После идет содержание сообщения, поясняющее произошедшее событие. После чего показаны оценки Appraisals и 
Feelings для доброжелательности, возбужденности и доминантности.

На (Рис. \ref{pic:ris11}) представлены гистограммы частот действий, совершаемых испытуемыми и модельным человеком.

\begin{figure}[h]
\includegraphics[width=0.75\columnwidth]{./img/ris11.png}
\centering
\caption{Диаграмма частот действий первого актора}
\label{pic:ris11}
\end{figure}

Исходя из диаграммы, можно заметить, что большое количество действий не использовалось. 
Данная ситуация возникла в ходе тестирования взаимодействия роботов с использованием моральной схемы, описанной в формуле \ref{eq:appraisals01}.

Существенным фактором, повлиявшим на работу моральной схемы, является оценка каждого действия отдельно по VAD, о чем говорися в работе \cite{Samsonovich02}. 
В ходе корректировки действий, основанной на опросе фокус группы, были произведены корректировки оценок. 
Результат корректировок виден на (Рис. \ref{pic:ris11}).

\section{Сбор данных, метрики и инструменты}

Для реализации сбора информации обычно используются следующие инструменты,  что отображено в работае \cite{parser01} :
\begin{itemize}
  \item	HTTP Client – клиент является библиотекой передачи, он находится на стороне клиента, отправляет и получает сообщения HTTP.
  \item	Beautiful Soup - это пакет Python для анализа документов HTML и XML. Он создает дерево синтаксического анализа для проанализированных страниц, которое можно использовать для извлечения данных из HTML, что полезно для парсинга веб-страниц.
  \item	Requests - это HTTP-библиотека для языка программирования Python. Цель проекта - сделать HTTP-запросы более простыми и удобными для человека.
  \item	AsyncIO – модуль, предназначенный для упрощения использования корутин и футур в асинхронном коде — чтобы код выглядел как синхронный, без коллбэков.
\end{itemize}

Основной принцип парсинга ресурсов для сбора информации осуществлялся при помощи написания алгоритмов на языке программирования Python. Для этого использовалась библиотека ButifulSoup, которая была выбрана для использования для чтения HTML, а также библиотека Requests, которая является стандартным инструментом для составления HTTP-запросов в Python. Простой и аккуратный API значительно облегчает трудоемкий процесс создания запросов. Таким образом, можно сосредоточиться на взаимодействии со службами и использовании данных в приложении. Так же из за большого количества реквестов было рационально использовать корутину, используя библиотеку Asyncio.

Такие HTTP методы, как GET и POST, определяют, какие действия будут выполнены при создании HTTP запроса. 
GET является одним из самых популярных HTTP методов. Метод GET указывает на то, что происходит попытка извлечь 
данные из определенного ресурса. Для того, чтобы выполнить запрос GET, используется requests.get(). 
Положительным ответом сервера является код 200. 

В метод requests.get() помещается URL, в которую может быть помещен JSON как текст запроса, либо при наличии 
API более упрощенный текстовый параметр, при наличии  токена как идентификатора аккаунта – токен.

response.status\_code, который равен 200, означает то что ответ от сервера получен положительный, 
текст запроса обработка, идентификация пользователя прошла успешна, возвращен ответ в формате JSON.

Иной же случай, когда нет доступа к API, оно платное либо его не существует. 
В таком случае скрипт будет выглядеть следующим образом.

В данном примере осуществляется request по URL, добавляется User-Agent, который имитирует пользователя, 
декодируются данные HTML в формат Unicode-8, который представляет собой распространённый стандарт 
кодирования символов, позволяющий более компактно хранить и передавать символы Юникода, используя 
переменное количество байт, и обеспечивающий полную обратную совместимость с 7-битной кодировкой ASCII.

Далее находим все классы таблиц и присваиваем переменным генератор, вытаскивающий 
из классов таблиц данные с соответствующим тегом.

С помощью BS4 можно найти в коде HTML все что требуется для создания нудного массива данных, 
далее для того, чтобы унифицировано хранить данные, следует использовать объектно-ориентированные форматы 
данных, такими бывают JSON и XML. JSON - текстовый формат обмена данными, основанный на JavaScript. 
Как и многие другие текстовые форматы, JSON легко читается людьми и в итоге был выбран как формат хранения данных в работе.

Для того чтобы спарсить сайт полностью и извлечь весь требуемый тип данных, используются рекурсивный подход. 
Для этого в работе используется пакет Python - html5lib, который реализует алгоритм парсинга HTML5, 
на который сильно влияют современные браузеры. Как только парсер получает нормализованную структуру 
содержимого, становится доступным поиск данных в любом дочернем элементе тега html. 
Искомые данные чаще всего находятся в теге “table”. После нахождения родительского тега, рекурсивно проходим по дочерним элементам. 
Для ссылок чаще всего используется тег “href”, далее из полученных ссылок извлекаем требуемый тип данных на сайте, в нашем случае это текст. 

Так как на сайте слишком много текста и достаточно большой процент содержания лишнего текста по типу маркировок, 
контекстной рекламы и прочего, ставится задача определения выявления значимого текста на сайте. 

Значимый текст выбирается на основании анализа его принадлежности к заранее определенным классам текста (или тематикам). 
Как правило, методы автоматической классификации основаны на методе машинного обучения: сначала получают 
обученную с помощью какого-либо алгоритма модель, качество которой определяет точность классификации. 
Таким образом, процесс обучения зависит от выбранного алгоритма и «чистоты» обучающей выборки, согласно работе о \cite{neural04}. 
Одним из фреймворков, который используется для того чтобы обучать модель по заранее определенным классам – это lingvo, 
реализованный для  .NET Framework, в котором используется Gradient Sign Dropout (GradDrop).

\section{Инструменты для анализа текста}

После сбора данных для их последующего использования в нейронной сети требуется осуществить разметку данных, для этого требуется 
выделить те слова определяющие какое-либо действие или предмет, которые будут семантически близки к заранее предопределенным 
типам действий или типам предметов.

Возможность идентификации семантической близости между словами сделала модель word2vec широко используемой в NLP-задачах, которые подробно описываются в \cite{neural05}. 
Идея word2vec основана на контекстной близости слов. Каждое слово может быть представлено в виде вектора, 
близкие координаты векторов могут быть интерпретированы как близкие по смыслу слова \cite{seman04}. 

Таким образом, извлечение семантических отношений (отношение синонимии, родовидовые отношения и другие) может быть автоматизировано. 
Установление семантических отношений вручную считается трудоемкой и необъективной задачей, требующей большого количества времени и 
привлечения экспертов. Но среди ассоциативных слов, сформированных с использованием модели word2vec, встречаются слова, не 
представляющие никаких отношений с главным словом, для которого был представлен ассоциативный ряд \cite{seman03}.

В работе рассматриваются дополнительные критерии, которые могут быть применимы для решения данной проблемы. Наблюдения и проведенные 
эксперименты с общеизвестными характеристиками, такими как частота слов, позиция в ассоциативном ряду, могут быть использованы для 
улучшения результатов при работе с векторным представлением слов в части определения семантических отношений для русского языка. 

Представление слов в виде векторов позволяет применять математические операции. В большинстве примеров можно встретить вычитание векторов, 
когда результат вычисления vec('Madrid') - vec('Spain') + vec('France') будет ближе к vec('Paris'), чем к другим векторам из распределения. 
Таким образом, разница векторов может быть использована для поиска семантических отношений между словами \cite{seman02}.

Word2vec не возвращает напрямую семантические отношения между словами. В ассоциативном ряду, который может быть возвращен 
в качестве близких слов к запрашиваемому (главному) слову, отражаются слова, которые часто употребляются рядом в контексте. 
Бесспорно, в ассоциативном ряду встречаются синонимы, антонимы, гипонимы, гиперонимы, холонимы, меронимы, ассоциации и другие 
типы, которые могут быть определены как семантические отношения.

\section{Диаграмма классов}

На (Рис. \ref{pic:ris12}) изображена диаграмма классов (Построение такой диаграммы рекомендуется в работе \cite{OOP}):

\begin{figure}[h]
\includegraphics[width=0.75\columnwidth]{./img/ris12.png}
\centering
\caption{Связь между акторами и объектами}
\label{pic:ris12}
\end{figure}

\begin{enumerate}
  \item	В Web Site Manager попадают ссылки на все сайты, каждому сайту указываются теги.
  \item	Далее Web Site Manager передает ссылки в Request Manager.
  \item	Request Manager асинхронно посылает запросы на сайт, чтобы получить весь контент сайта, по запросу он получает html-страницу и передает ее в Web Html Parser.
  \item	Web Html Parser по тегам парсит html так, что выделяет нужный текст, который передается в Web Text.
  \item	Context Text Analisator -  использует консольное приложение по работе с LingvoNET, с которым класс общается посредством стандартного ввода/вывода с целью классификации текста на предмет семантической принадлежности.
  \item	В Web Text выделяются ключевые слова при помощи Virtual Actor Action.
  \item	Virtual Actor Action берутся из excel-таблицы действий со значением весов и описаниями.
  \item	В Data Holder на основании Virtual Actor Action и самого текста генерируются сценарии (10 действий, 11 состояний), моделью пытаемся предсказать 11 состояние на основе 10 предыдущих.
  \item	Actor Behavior Network хранит все обученные модели, из которых потом выбирается оптимальная.
  \item	Network Factory – шаблон проектирования, создает по описанию нейронной сети создает экземпляр сети (Actor Behavior Network).
  \item	Одному Virtual Actor соответствует одна нейронная сеть, у Virtual Actor меняется текущее состояние на основе модели. 
\end{enumerate}

\section{Выводы}

В процессе работы были выделены часто используемые инструменты для сбора данных со сторонних веб-сервисов, 
а именно программные модули языка Python requests, Asyncio, ButifulSoup. 
Произведен анализ инструментария для решения задач семантической близости слов. 
Также была составлена диаграмма классов программного модуля. 
А для моделирования поведения виртуального агента составлена архитектура рекуррентной нейронной сети.





\clearpage

\chapter{Реализация программного продукта}

В этой главе описаны практические методы и их частичное представлении реализации, 
которые мы использовали для достижения поставленной цели.

\section{Использование парсера данных}

\clearpage

\chapter*{Заключение}
\addcontentsline{toc}{chapter}{Заключение}

В рамках представленной работы были изучены проблемы парсинга сайтов и классические методы их решения,
были выделены основные нейросетевые подходы для анализа текстов, выделены основные проблемы связанные 
с семантитеческой классификацией, а так же изучены библиотеки и плагины, которые применяются для реализации данных подходов. 
Был применен метод опорных векторов (или SVM – Support Vector Machine) 
для определения семантитеческой принадлежности текста к юмористическимо сюжету, а так же, 
оптимизирован алгоритм поиска требуемых данных на юмористических ресурсах. 

По заврешению получения юмористических данных в процессе парсинга, были получены данные, которые 
в последствии были размечены по принципу определения близости слов с заранее описанными действиями
для виртуальных акторов.

Размеченные данные были использованы в последствии для обучения модели задача которой генерировать 
юмористические сюжеты, что подтверждается результатами обучения.
\begin{enumerate}
    \item Были определены подходы, согласно котором разрабатывалась альтернативная линия сюжетов и их воплощения Виртуальным Актором.
    \item Был осуществлен сбор и осуществлена разметкуа данных для обучения нейронной сети;
    \item Был осуществлен глубокий интеллектуальный анализ данных по размеченным массивам данных.
    \item Был разработан, алгоритм передающей результаты анализа виртуальному Агенту.
    \item Было Реализован визуальный агент и сцена, используя межплатформенную среду разработки компьютерных игр Unity3d.
\end{enumerate}



%Цель работы была полностью достигнута, и поставленные задачи решены. Результаты
%выполнения ВКР были использованы в программном обеспечении, разрабатываемом на
%предприятии АО "Концерн "Созвездие".


\clearpage

\input{chapters/thesis-template-bibl.tex}

\endrefsection

\clearpage

%\chapter*{Приложения}
%\addcontentsline{toc}{chapter}{Приложения}
%\appendixtocon
%\renewcommand{\appendixname}{Приложение}
\appendix
\renewcommand{\appendixtocname}{Приложения}
\addappheadtotoc
%\titleformat{\chapter}[block]{\centering\normalfont\Large\bfseries}{\chaptername{} \thechapter.}{1ex}{}{}
\renewcommand{\chaptername}{Приложение}
%\renewcommand*\printchaptername{\Large\bfseries\appendixname~}
%\renewcommand{\thechapter}{Приложение \Alph{chapter}}
%\renewcommand{\thechaptertoc}{Приложение \Alph{chapter}}

%\renewcommand{\chaptermark}[1]{\markboth{\chaptername\ \thechapter.\ #1}{}}

%\begin{appendices}

%!!!
%\input{chapters/thesis-template-appendix1.tex}

%\clearpage

%\input{chapters/thesis-template-appendix2.tex}

%\clearpage

%\chapter{Правила использования шаблона}\label{app-manual}

Настоящий шаблон все еще несколько несовершенен в плане оформления: например, неправильная нумерация приложений, и еще несколько нюансов. В последующих версиях это будет исправляться.

Ниже описана структура исходных текстов шаблона (и, соответственно, структура исходных текстов ПЗ).

Кодировка всех файлов — UTF8, и для сборки PDF документов следует использовать команду \texttt{xelatex}.

Головной файл --- \texttt{thesis-template.tex}. Его задача --- <<склеить>> вместе разные части ПЗ. Каждая часть (реферат, введение, каждая содержательная глава, заключение, библиография, приложения) выделяется в отдельный файл. 

\begin{itemize}
  \item[] \texttt{thesis-macro.tex} --- содержит определения различных макрокоманд, которые часто используются в конкретной работе, например, определения окружения для теорем, некоторые часто используемые формулы, и т.п.;
  \item[] \texttt{thesis-abstract.tex} --- содержит аннотацию;
  \item[] \texttt{thesis-intro.tex} --- содержит введение;
  \item[] \texttt{thesis-chapter1.tex} --- текст первой главы;
  \item[] \texttt{thesis-chapter2.tex} --- текст второй главы;
  \item[] \texttt{thesis-chapter3.tex} --- текст третьей главы;
  \item[] \texttt{thesis-bibl.tex} --- список литературы (только подключение к
  проекту);
  \item[] \texttt{biblio.bib} --- собственно библиография (в формате BibTeX);
  \item[] \texttt{thesis-conclusion.tex} --- заключение;
  \item[] \texttt{thesis-appendix1.tex} --- первое приложение;
  \item[] \texttt{thesis-appendix2.tex} --- второе приложение;
\end{itemize}


%Одна из первых вещей, которые необходимо сделать при использовании данного шаблона --- это отредактировать аргумент команды \verb|\headertext| в начале головного файла.

Головной файл нужно менять лишь тогда, когда нужно добавить в проект новый файл,
или удалить существующий (см. команду \verb|\input|). Обычно, это требуется,
когда нужно добавить/удалить приложения.

\section{Титульные листы}

Существует два варианта генерации титульных листов:

\begin{itemize}
  \item использование листов, сверстанных в \LaTeX{} (используется по
  умолчанию);
  \item подстановка пустых бланков из PDF-файлов.
\end{itemize}

\subsection{Титульные листы LaTeX}

Проект содержит определения титульных листов, описанные в виде \textbf{.tex}
файлов. На данный момент требуется заполнение данных студента вручную. Позже
будет реализована автоматическая подстановка данных проекта при инициализации
репозитория.

\subsection{Титульные листы из PDF}

При возникновении проблем с использованием титульных листов \LaTeX{} возмодно
включить в документ бланки титульных листов из PDF-файлов. Для этого нужно
раскомментировать соответствующие команды \textbf{includepdf} в начале
документа.

Для того, чтобы \LaTeX{} при компиляции автоматически <<подхватил>> задание, его
нужно сохранить в формате pdf (например, с помощью вирутального принтера),
поместить в ту же папку \texttt{/title} и назвать \texttt{task.pdf}. Точно также
следует поступить с титульной страницей (\texttt{title.pdf}). При оформлении ПЗ
для ВКР следует дополнительно поместить в папку \texttt{/title} pdf-версию листа
с подписями, назвав файл \texttt{title-dep22.pdf}. После этого нужно
раскомментировать команду
\begin{center}
  \verb|\includepdf[ ... ]{title/title-dep22.pdf}|
\end{center}
в начале головного файла.

Образцы и Word-шаблоны титульных листов для (РС)ПЗ к УИРам, НИРам, практикам и
ВКР доступны в репозитории
\begin{center}
  \url{https://gitlab.com/skibcsit/thesis-titles}.
\end{center}  

\textbf{Замечание}. В шаблоне используется пакет \texttt{hyperref}, который
делает две вещи: все перекрестные ссылки <<кликабельны>>, а также выделены
(красной) рамочкой. Эти рамки \textit{не выводятся на печать}. Вместо цветных
рамок, возможны другие способы выделения ссылок (см. документацию пакета).

%!!!

%\end{appendices}

\end{document}



