\chapter*{Реферат}
\thispagestyle{plain}


Пояснительная записка содержит 46 страниц, 19 рисунков, ~-- 32 источников литературы.

Ключевые слова: UNITY3D, EBICA, СОЦИАЛЬНО-ЭМОЦИОНАЛЬНЫЙ ИНТЕЛЛЕКТ, ВИРТУАЛЬНЫЙ АКТОР, МАШИННОЕ ОБУЧЕНИЕ, РЕККУРЕНТНЫЕ СЕТИ 

Объектом исследования являются экспертные системы.

Предмет исследования - модель когнитивной архитектуры для создания Виртуального Актора.

Целью данной научно-исследовательской работы является создание прототипа Виртуального Актора с многомодальным интерфейсом, 
модель интеллекта которого основана на когнитивной архитектуре eBICA.
Существует два глобальных подхода к созданию социально-эмоционального интеллекта. 
Один основанный на нейросетях, другой на когнитивных архитектурах.
В ходе работы над НИР был разработан и протестирован с участием испытуемых прототип Виртуального Актора, 
обладающий социально-эмоциональным интеллектом и помещенный в виртуальное окружение, которое создано при помощи графического движка Unity3d, 
была реализована система для анализа речи с выявлением эмоций соответствующим речевым признакам, так же были
в дополнении к вышеуказанной системе, были спроектированы и реализованы методы 
воздействия на виртуального агента, основывающиеся на семантической составляющей речевого контекста.
Данная работа является актуальной поскольку на данный момент эта область находится на начальных этапах 
развития и активной интеграции в различные индустрии.
Созданная и протестированная модель интеллекта затем может быть интегрирована в другие проекты с 
Виртуальным Актором: виртуальный слушатель, виртуальный клоун, виртуальный танцор. 