\chapter*{Реферат}
\thispagestyle{plain}


Пояснительная записка содержит - страниц, - рисунков, -  источников литературы.

Ключевые слова: UNITY3D, EBICA, СОЦИАЛЬНО-ЭМОЦИОНАЛЬНЫЙ ИНТЕЛЛЕКТ, ВИРТУАЛЬНЫЙ АКТОР, МАШИННОЕ ОБУЧЕНИЕ, РЕККУРЕНТНЫЕ СЕТИ 

Объектом исследования в данной работе является социально-эмоциональный интеллект

Предмет исследования в данной работе - модель когнитивной архитектуры для создания Виртуального Актора, 
а так же подходы машинного обучения расширяющие его функционал

В первом разделе обозначены когнитивные архитектуры, выявляются их преимущества и недостатки по 
сравнению с когнитивной архитектурой eBICA. Определяются типы распознавания и синтеза речи. 
Выделятся классификация эмоций.
Рассматриваются методы распознавания эмоций в тексте.
Ставятся цели и задачи научно-исследовательской работы. 

Во втором производится анализ когнитивных архитектур и производится анализ методов распознавания человеческой речи,
анализ подходов в машинном обучении для распознавания эмоций речи.
Так же приводятся теоретические выкладки описания модели социально-эмоционального интеллекта. 

В третьем разделе приводится подробное описание работы алгоритма, реализующего когнитивную модель социально-эмоционального интеллекта с учетом
интегрированной системы распознавания речи и распознавания эмоций в нейю
Строятся блок-схемы для кодовой реализации модели интеллекта.

В четвертом разделе приводятся реализация программного продукта представляющего собой когнитивно эмоциональный интеллект, 
интегрированную систему распознавания речи и выделения эмотивных признаков, показаны результаты в иллюстрациях.
а так же приведены способный взаимодействовать на эмоциональной основе.