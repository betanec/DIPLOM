\chapter*{Введение}
\label{sec:afterwords}
\addcontentsline{toc}{chapter}{Введение}


В последнее время все большую и большую популярность набирают технологии предоставляющие возможность участвовать человеку 
в виртуальном мире либо технические средства, позволяющие представление виртуальную реальность в реальном мире.
Виртуальные Акторы способны в будущем заменить докладчика на конференциях различного характера и представлениях,
быть интегрированы в процесс обучения студентов, быть использользованы как развивающие игрушки для детей.
Однако, до сих пор не существует реализованной модели, которая представляет из себя полноценный эмоциональный интеллект.

Вдобавок, даже в настоящем взайимодействие человека и машины становится буквально повседневным и может во многом
в будущем упростить жизнь людей. К примеру примненением для системы взаимодействия робота и человека может быть 
выражено посредсвом их вербального общения, такое взаимодействие может быть полезно для оказания помощи немощным людям. 
Технология распознающая эмоции может отслеживать эмоциональное состояние человека для выявления аномалий в его поведении. 
Когда возникает аномалия, это может означать, что человек требует внимания.

Кроме того, распознавание эмоций может быть практичным в диагностике некоторых заболеваний 
(депрессивные расстройства, болезнь Паркинсона,
и т.д.) 
выявлением дефицита выражения определенных эмоций, ускорением диагноз, так и лечение потенциального пациента.


Целью данной научно-исследовательской работы является создание прототипа Виртуального Актора с многомодальным интерфейсом, 
модель интеллекта которого основана на когнитивной архитектуре eBICA.
Существует два глобальных подхода к созданию социально-эмоционального интеллекта. 
Один основанный на нейросетях, другой на когнитивных архитектурах.
В ходе работы над ВКР был разработан и протестирован прототип Виртуального Актора, 
обладающий социально-эмоциональным интеллектом и помещенный в виртуальное окружение, которое создано при помощи графического движка Unity3d, 
была реализована система для анализа речи с выявлением эмоций соответствующим речевым признакам, так же были
в дополнении к вышеуказанной системе, были спроектированы и реализованы методы 
воздействия на виртуального агента, основывающиеся на семантической составляющей речевого контекста.
Данная работа является актуальной поскольку на данный момент эта область находится на начальных этапах 
развития и активной интеграции в различные индустрии.
Созданная и протестированная модель интеллекта затем может быть интегрирована в другие проекты с 
Виртуальным Актором: виртуальный слушатель, виртуальный клоун, виртуальный танцор. 

