\chapter*{Введение}
\label{sec:afterwords}
\addcontentsline{toc}{chapter}{Введение}


В последнее время все большую и большую популярность набирают технологии предоставляющие возможность участвовать человеку 
в виртуальном мире либо технические средства, позволяющие представление виртуальную реальность в реальном мире. 
Виртуальные Акторы способны в будущем заменить докладчика на конференциях различного характера и представлениях.

Целью исследования является создание общей вычислительной модели механизмов, лежащих в основе человеческих эмоций. Осуществляется это путем создания Виртуального агента, подкрепленного когнитивной архитектурой и помещенного в виртуальное окружение. В данной парадигме человек (испытуемый, проводящий сеанс игры) может взаимодействовать с Виртуальным Актором, воплощенного в виде аватара в игре с трехмерной графикой, и оценивать его по различным параметрам. Такая модель может интерпретировать человеческое поведение и на основе экспериментов можно делать выводы о ее социальной приемлемости и точности имитирования человеческих эмоций.

В первом разделе проводится анализ существующих когнитивных архитектур, выявляются их преимущества и недостатки по 
сравнению с когнитивной архитектурой eBICA. Ставятся цели и задачи научно-исследовательской работы. 

Во втором разделе приводятся теоретические выкладки описания модели социально-эмоционального интеллекта. 
Формируется список тестов для проверки возможностей данной модели. Описание методов логирования и сбора данных игровых сессий испытуемых.

В третьем разделе приводится подробное описание работы алгоритма, реализующего когнитивную модель социально-эмоционального интеллекта.
Строятся блок-схемы для кодовой реализации модели интеллекта.

В четвертом разделе приводятся и анализируются программные средства для реализации полученной парадигмы.