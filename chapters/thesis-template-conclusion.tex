\chapter*{Заключение}
\addcontentsline{toc}{chapter}{Заключение}

В рамках представленной работы были изучены проблемы парсинга сайтов и классические методы их решения,
были выделены основные нейросетевые подходы для анализа текстов, выделены основные проблемы связанные 
с семантитеческой классификацией, а так же изучены библиотеки и плагины, которые применяются для реализации данных подходов. 
Был применен метод опорных векторов (или SVM – Support Vector Machine) 
для определения семантитеческой принадлежности текста к юмористическимо сюжету, а так же, 
оптимизирован алгоритм поиска требуемых данных на юмористических ресурсах. 

По заврешению получения юмористических данных в процессе парсинга, были получены данные, которые 
в последствии были размечены по принципу определения близости слов с заранее описанными действиями
для виртуальных акторов.

Размеченные данные были использованы в последствии для обучения модели задача которой генерировать 
юмористические сюжеты, что подтверждается результатами обучения.
\begin{enumerate}
    \item Были определены подходы, согласно котором разрабатывалась альтернативная линия сюжетов и их воплощения Виртуальным Актором.
    \item Был осуществлен сбор и осуществлена разметкуа данных для обучения нейронной сети;
    \item Был осуществлен глубокий интеллектуальный анализ данных по размеченным массивам данных.
    \item Был разработан, алгоритм передающей результаты анализа виртуальному Агенту.
    \item Было Реализован визуальный агент и сцена, используя межплатформенную среду разработки компьютерных игр Unity3d.
\end{enumerate}



%Цель работы была полностью достигнута, и поставленные задачи решены. Результаты
%выполнения ВКР были использованы в программном обеспечении, разрабатываемом на
%предприятии АО "Концерн "Созвездие".
