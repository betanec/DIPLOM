\chapter*{Заключение}
\addcontentsline{toc}{chapter}{Заключение}

В ходе работы над НИР был разработан и протестирован с участием испытуемых прототип Виртуального Актора, 
обладающий социально-эмоциональным интеллектом и помещенный в виртуальное окружение, которое создано при помощи графического движка Unity3d, 
была реализована система для анализа речи с выявлением эмоций соответствующим речевым признакам, так же были
в дополнении к вышеуказанной системе, были спроектированы и реализованы методы 
воздействия на виртуального агента, основывающиеся на семантической составляющей речевого контекста.
Данная работа является актуальной поскольку на данный момент эта область находится на начальных этапах 
развития и активной интеграции в различные индустрии.
Созданная и протестированная модель интеллекта затем может быть интегрирована в другие проекты с 
Виртуальным Актором: виртуальный слушатель, виртуальный клоун, виртуальный танцор. 

\begin{enumerate}
    \item Были проанализированы различные когнитивные архитектуры.
    \item Проанализированы методы распознавания речи.
    \item Доработан алгоритм поведения Виртуального Актора.
    \item Унифицирован алгоритм поведения Виртуального Актора.
    \item Был разработан, алгоритм выявления эмоций из человеческой речи.
    \item Добавлено возможность эмоционального взаимодействия с виртуальным актором.
    \item Было Реализован визуальный агент и сцена, используя межплатформенную среду разработки компьютерных игр Unity3d.
\end{enumerate}



%Цель работы была полностью достигнута, и поставленные задачи решены. Результаты
%выполнения ВКР были использованы в программном обеспечении, разрабатываемом на
%предприятии АО "Концерн "Созвездие".
